% TASNI Paper - Introduction
% Thermal Anomaly Search for Non-communicating Intelligence

\section{Introduction}
\label{sec:introduction}

\subsection{Motivation: Searching for Thermal Anomalies}

The identification of unusual astrophysical sources has historically driven major discoveries, from quasars \citep{Schmidt1963} to gamma-ray bursts \citep{Klebesadel1973}. In the modern era of large-area sky surveys, systematic searches for anomalous objects---those that defy easy classification---offer a promising avenue for discovering new phenomena.

One class of potentially anomalous sources comprises objects that emit primarily in the thermal infrared while remaining undetected at other wavelengths. Such ``thermal orphans'' could arise from several physical mechanisms:

\begin{enumerate}
    \item \textbf{Extremely cold brown dwarfs}: Objects with $T_{\rm eff} \lesssim 300$~K emit predominantly at wavelengths $\lambda > 10~\mu$m, with negligible optical flux. The coldest known brown dwarf, WISE~J085510.83$-$071442.5, has $T_{\rm eff} \approx 250$~K \citep{Luhman2014} and is detectable only in the mid-infrared.

    \item \textbf{Dust-obscured sources}: Objects embedded in optically thick dust shells re-radiate absorbed energy as thermal emission at temperatures set by the dust sublimation radius.

    \item \textbf{Technosignatures}: Theoretical considerations suggest that advanced technological civilizations might be detectable through their waste heat \citep{Dyson1960, Kardashev1964}. A structure intercepting stellar luminosity would re-radiate at temperatures $T \sim 300$~K for Sun-like stars at 1~AU separation \citep{Wright2014a, Wright2014b}.
\end{enumerate}

While the first two explanations invoke known astrophysics, the third---though speculative---motivates careful characterization of any genuinely anomalous thermal sources. Even if all candidates prove to be natural objects, systematic searches constrain the prevalence of technological activity in the solar neighborhood.

\subsection{The Y Dwarf Population}

Brown dwarfs are substellar objects with masses below the hydrogen-burning limit ($\sim$0.075~$M_\odot$). They cool continuously throughout their lifetimes, passing through spectral types M, L, T, and Y as their effective temperatures decline \citep{Kirkpatrick2005, Cushing2011}.

The Y dwarf spectral class, defined by $T_{\rm eff} \lesssim 500$~K, represents the coldest end of the brown dwarf sequence \citep{Cushing2011, Kirkpatrick2012}. These objects are characterized by:

\begin{itemize}
    \item Strong methane (CH$_4$) and water (H$_2$O) absorption in the near-infrared
    \item Extremely red mid-infrared colors (W1$-$W2 $> 2$~mag)
    \item Ammonia (NH$_3$) features in the coldest examples
    \item Negligible optical emission ($M_V > 25$~mag)
\end{itemize}

Approximately 30 Y dwarfs are currently known, identified primarily through WISE color selection \citep{Kirkpatrick2012, Kirkpatrick2021}. The space density of Y dwarfs remains poorly constrained due to the difficulty of detecting these intrinsically faint objects. Population synthesis models predict a substantial population of cold ($T < 300$~K) brown dwarfs in the solar neighborhood awaiting discovery \citep{Burgasser2004, Ryan2017}.

\subsection{Wide-field Infrared Survey Explorer}

The Wide-field Infrared Survey Explorer (WISE; \citealt{Wright2010}) mapped the entire sky in four mid-infrared bands: W1 (3.4~$\mu$m), W2 (4.6~$\mu$m), W3 (12~$\mu$m), and W4 (22~$\mu$m). The AllWISE data release \citep{Cutri2013} contains 747 million sources, providing an unprecedented census of mid-infrared emitters.

For cold objects ($T \lesssim 500$~K), the W1$-$W2 color provides a sensitive temperature diagnostic. The Rayleigh-Jeans tail of the Planck function produces increasingly red colors as temperature decreases:

\begin{equation}
\text{W1} - \text{W2} \approx 2.5 \log_{10}\left(\frac{B_\nu(4.6~\mu\text{m}, T)}{B_\nu(3.4~\mu\text{m}, T)}\right)
\end{equation}

For $T = 300$~K, this yields W1$-$W2 $\approx 2$~mag, while $T = 250$~K produces W1$-$W2 $\approx 3$~mag. These extreme colors distinguish cold thermal emitters from the stellar locus.

The NEOWISE reactivation mission \citep{Mainzer2014} has continued W1 and W2 observations since 2013, providing a decade-long temporal baseline for variability studies. This extended coverage enables identification of sources with unusual long-term photometric behavior.

\subsection{Multi-Wavelength Veto Strategy}

Previous searches for ultracool dwarfs have relied primarily on color selection, identifying red objects in WISE photometry \citep{Kirkpatrick2011, Kirkpatrick2012, Cushing2011}. While effective, this approach cannot distinguish genuinely ``invisible'' objects from those with faint but detectable counterparts at other wavelengths.

We adopt a complementary strategy based on multi-wavelength non-detection. By requiring absence from optical (Gaia, Pan-STARRS, Legacy Survey), near-infrared (2MASS), and radio (NVSS) catalogs, we isolate objects detectable \emph{only} in the mid-infrared. This ``veto'' approach has several advantages:

\begin{enumerate}
    \item \textbf{Purity}: Sources passing all vetoes are guaranteed to lack counterparts above survey detection limits, ensuring genuine optical/NIR invisibility.

    \item \textbf{Completeness}: Unlike color cuts, the veto strategy does not depend on assumed spectral energy distribution shapes.

    \item \textbf{Anomaly detection}: By selecting against known source classes (stars, galaxies, AGN), the veto strategy naturally identifies unusual objects.
\end{enumerate}

The combination of thermal colors and multi-wavelength invisibility defines our target population: room-temperature emitters with no detectable emission at shorter or longer wavelengths.

\subsection{This Work}

We present the Thermal Anomaly Search for Non-communicating Intelligence (TASNI), a systematic pipeline to identify mid-infrared sources lacking counterparts across the electromagnetic spectrum. Our goals are:

\begin{enumerate}
    \item To quantify the population of genuinely ``invisible'' thermal emitters in the AllWISE catalog

    \item To characterize their photometric, kinematic, and temporal properties

    \item To identify candidates for spectroscopic follow-up that may represent the coldest brown dwarfs or other unusual objects

    \item To constrain the prevalence of anomalous thermal signatures in the solar neighborhood
\end{enumerate}

The paper is organized as follows. Section~\ref{sec:methods} describes our data sources, selection criteria, and analysis methodology. Section~\ref{sec:results} presents the pipeline results, including the discovery of four ``fading thermal orphans'' with unprecedented properties. Section~\ref{sec:discussion} discusses the physical interpretation of these sources and constraints on alternative hypotheses. Section~\ref{sec:conclusions} summarizes our conclusions and outlines future observations.

\end{document}
