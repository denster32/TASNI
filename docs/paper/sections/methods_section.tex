% TASNI Paper - Methods Section
% Thermal Anomaly Search for Non-communicating Intelligence

\section{Methods}
\label{sec:methods}

We developed the Thermal Anomaly Search for Non-communicating Intelligence (TASNI) pipeline to systematically identify infrared-bright sources that lack counterparts across the electromagnetic spectrum. Our approach leverages multi-wavelength archival data to isolate objects detectable only in the mid-infrared, then applies temporal analysis to characterize their long-term behavior. Below we describe our data sources, selection criteria, cross-matching strategy, and variability analysis methodology.

\subsection{Data Sources}
\label{sec:data_sources}

\subsubsection{Primary Infrared Catalog}

Our parent sample is drawn from the AllWISE Source Catalog \citep{Wright2010, Cutri2013}, which contains 747,634,026 sources observed by the Wide-field Infrared Survey Explorer (WISE) during its cryogenic and post-cryogenic mission phases (2010--2011). AllWISE provides photometry in four mid-infrared bands: W1 (3.4~$\mu$m), W2 (4.6~$\mu$m), W3 (12~$\mu$m), and W4 (22~$\mu$m), with 5$\sigma$ point source sensitivities of 0.068, 0.098, 0.86, and 5.4~mJy, respectively.

For temporal analysis, we utilize the NEOWISE Reactivation Single-Exposure Source Table \citep{Mainzer2014}, which provides multi-epoch photometry from 2013 to present. The combination of AllWISE and NEOWISE-R data enables variability analysis over a $\sim$10-year baseline with typically 250--500 epochs per source at 6-month cadence.

\subsubsection{Optical and Near-Infrared Veto Catalogs}

To identify sources lacking optical/NIR counterparts, we cross-match against:

\begin{itemize}
    \item \textbf{Gaia DR3} \citep{GaiaCollaboration2023}: 1.8 billion sources with $G < 21$~mag, providing optical photometry and astrometry. We use Gaia proper motions when available.

    \item \textbf{2MASS Point Source Catalog} \citep{Skrutskie2006}: 470 million sources with $J$, $H$, $K_s$ photometry to $K_s \approx 14.3$~mag.

    \item \textbf{Pan-STARRS DR1} \citep{Chambers2016}: 3 billion detections in $grizy$ bands covering the sky north of $\delta = -30°$ to depths of $g \approx 23.3$, $r \approx 23.2$~mag.

    \item \textbf{DESI Legacy Imaging Surveys DR10} \citep{Dey2019}: Deep optical imaging ($g \approx 24.7$, $r \approx 23.9$, $z \approx 23.0$~mag) covering 14,000~deg$^2$.
\end{itemize}

\subsubsection{Radio and X-ray Veto Catalogs}

To exclude active galactic nuclei (AGN) and other high-energy sources:

\begin{itemize}
    \item \textbf{NVSS} \citep{Condon1998}: 1.4~GHz radio survey covering $\delta > -40°$ with rms $\approx 0.45$~mJy/beam.

    \item \textbf{ROSAT All-Sky Survey} \citep{Voges1999}: X-ray catalog (0.1--2.4~keV) for excluding coronal emitters and AGN.
\end{itemize}

\subsubsection{Spectroscopic Catalogs}

We cross-match against spectroscopic surveys to identify previously classified objects:

\begin{itemize}
    \item \textbf{LAMOST DR7} \citep{Cui2012}: Low-resolution ($R \sim 1800$) optical spectra for 10+ million sources, providing stellar classifications and radial velocities.

    \item \textbf{SIMBAD} \citep{Wenger2000}: Astronomical database for identifying known objects.
\end{itemize}

\subsection{Source Selection}
\label{sec:selection}

Our selection pipeline applies successive filters to isolate thermally anomalous sources (Figure~\ref{fig:flowchart}):

\subsubsection{Tier 1: Optical Invisibility}

We first select AllWISE sources lacking Gaia DR3 counterparts within 3\arcsec:
\begin{equation}
    N_{\rm Gaia}(r < 3\arcsec) = 0
\end{equation}
This removes 341 million optically bright sources, yielding 406,387,755 candidates.

\subsubsection{Tier 2: Thermal Color Selection}

We apply a color cut to select sources with thermal (blackbody-like) spectral energy distributions:
\begin{equation}
    \mathrm{W1} - \mathrm{W2} > 0.5~\mathrm{mag}
\end{equation}
This criterion selects objects with effective temperatures $T_{\rm eff} \lesssim 1000$~K, consistent with cool brown dwarfs or room-temperature thermal emitters. Sources with W1$-$W2 $> 2$~mag correspond to $T_{\rm eff} \lesssim 400$~K.

\subsubsection{Tier 3: Near-Infrared Invisibility}

We remove sources with 2MASS counterparts within 3\arcsec:
\begin{equation}
    N_{\rm 2MASS}(r < 3\arcsec) = 0
\end{equation}
This eliminates late-type stars and most L/T dwarfs detectable in the near-infrared, reducing the sample to 62,856 candidates.

\subsubsection{Tier 4: Deep Optical Invisibility}

We further require non-detection in Pan-STARRS DR1 (where available) and Legacy Survey DR10:
\begin{equation}
    N_{\rm PS1}(r < 3\arcsec) = 0 \quad \mathrm{and} \quad N_{\rm Legacy}(r < 3\arcsec) = 0
\end{equation}
After this stage, 39,151 sources remain---objects detectable \emph{only} in mid-infrared wavelengths.

\subsubsection{Tier 5: Radio Silence}

To exclude AGN and radio-loud sources, we require:
\begin{equation}
    N_{\rm NVSS}(r < 30\arcsec) = 0
\end{equation}
This yields 4,137 radio-silent thermal anomaly candidates.

\subsubsection{Golden Sample Selection}

From the Tier 5 sample, we select the 100 highest-scoring candidates based on a composite anomaly score (Section~\ref{sec:scoring}). These ``golden targets'' are prioritized for detailed analysis and follow-up observations.

\subsection{Cross-Matching Methodology}
\label{sec:crossmatch}

All catalog cross-matches are performed using positional coincidence with radius $r_{\rm match}$:
\begin{equation}
    \Delta\theta = \sqrt{(\alpha_1 - \alpha_2)^2 \cos^2\delta + (\delta_1 - \delta_2)^2} < r_{\rm match}
\end{equation}
where $\alpha$ and $\delta$ are right ascension and declination. We adopt $r_{\rm match} = 3\arcsec$ for optical/NIR catalogs (accounting for WISE positional uncertainty of $\sim$0.5\arcsec\ and proper motion over the $\sim$10-year baseline) and $r_{\rm match} = 30\arcsec$ for radio catalogs (reflecting the larger NVSS beam).

For sources with Gaia proper motions, we propagate positions to the AllWISE epoch (2010.5) before cross-matching to account for source motion.

\subsection{Physical Parameter Estimation}
\label{sec:parameters}

\subsubsection{Effective Temperature}

We estimate effective temperatures by fitting a single-temperature blackbody to the W1 and W2 photometry:
\begin{equation}
    T_{\rm eff} = \frac{hc}{k_B} \left[ \frac{1}{\lambda_{\rm W2}} - \frac{1}{\lambda_{\rm W1}} \right] \left[ \ln\left(\frac{F_{\rm W1}}{F_{\rm W2}} \cdot \frac{\lambda_{\rm W1}^5}{\lambda_{\rm W2}^5}\right) \right]^{-1}
\end{equation}
where $\lambda_{\rm W1} = 3.4~\mu$m and $\lambda_{\rm W2} = 4.6~\mu$m. For our golden sample, we find $\langle T_{\rm eff} \rangle = 265 \pm 36$~K, with a range of 200--500~K.

\subsubsection{Distance Estimates}

For sources with measured proper motions $\mu$, we estimate distances assuming a typical tangential velocity $v_\perp$:
\begin{equation}
    d = \frac{v_\perp}{4.74 \cdot \mu}~\mathrm{pc}
\end{equation}
where $\mu$ is in mas~yr$^{-1}$ and $v_\perp$ is in km~s$^{-1}$. Adopting $v_\perp = 30$~km~s$^{-1}$ (typical for disk objects), our golden sample spans estimated distances of 18--115~pc, with $\langle d \rangle \approx 50$~pc.

\subsection{Variability Analysis}
\label{sec:variability}

\subsubsection{NEOWISE Light Curve Extraction}

We query the NEOWISE Reactivation Single-Exposure Source Table via the IRSA TAP service for each golden target within a 3\arcsec\ cone. This yields multi-epoch W1 and W2 photometry spanning MJD 56639--60523 (2013.9--2024.5), with a mean of 387 epochs per source over a 9.2-year baseline.

\subsubsection{Variability Metrics}

For each source, we compute:

\begin{enumerate}
    \item \textbf{Root-mean-square (RMS) variability}:
    \begin{equation}
        \sigma_{\rm rms} = \sqrt{\frac{1}{N-1} \sum_{i=1}^{N} (m_i - \bar{m})^2}
    \end{equation}

    \item \textbf{Reduced chi-squared}:
    \begin{equation}
        \chi^2_\nu = \frac{1}{N-1} \sum_{i=1}^{N} \frac{(m_i - \bar{m})^2}{\sigma_i^2}
    \end{equation}
    where $\sigma_i$ is the photometric uncertainty for epoch $i$.

    \item \textbf{Stetson J index} \citep{Stetson1996}:
    \begin{equation}
        J = \frac{\sum_{k} w_k \cdot \mathrm{sgn}(P_k) \sqrt{|P_k|}}{\sum_k w_k}
    \end{equation}
    where $P_k = \delta_{i(k)} \delta_{j(k)}$ is the product of normalized residuals for paired observations.

    \item \textbf{Linear trend slope}:
    \begin{equation}
        \frac{dm}{dt} = \frac{\sum_i (t_i - \bar{t})(m_i - \bar{m})}{\sum_i (t_i - \bar{t})^2}
    \end{equation}
    with significance assessed via the $p$-value of the linear regression.
\end{enumerate}

\subsubsection{Variability Classification}

Sources are classified as:
\begin{itemize}
    \item \textbf{VARIABLE}: $\chi^2_\nu > 3$ in either band, indicating significant scatter beyond photometric uncertainties.

    \item \textbf{FADING}: Significant negative trend ($p < 0.01$) with $dm/dt > 15$~mmag~yr$^{-1}$, indicating systematic dimming.

    \item \textbf{NORMAL}: Neither variable nor fading; consistent with stable emission.
\end{itemize}

Of our 100 golden targets, we classify 45 (45\%) as NORMAL, 50 (50\%) as VARIABLE, and 5 (5\%) as FADING.

\subsubsection{Periodogram Analysis}

We search for periodic signals using the Lomb-Scargle periodogram \citep{Lomb1976, Scargle1982} implemented in \texttt{astropy.timeseries}. We sample 10,000 trial periods logarithmically spaced from 0.5 to 1000 days and compute the false alarm probability (FAP) for the highest peak. We consider periods significant if FAP $< 0.01$.

We note that the NEOWISE observing cadence (approximately 6-month intervals) introduces aliasing at periods near 180 days. Periods in this range should be interpreted with caution.

\subsection{Anomaly Scoring}
\label{sec:scoring}

We assign each candidate a composite anomaly score based on multiple criteria:

\begin{equation}
    S_{\rm total} = S_{\rm stealth} + S_{\rm thermal} + S_{\rm kinematic} + S_{\rm variability}
\end{equation}

\subsubsection{Stealth Score}

Points are awarded for non-detection in each wavelength regime:
\begin{align}
    S_{\rm stealth} &= 10 \times N_{\rm veto}
\end{align}
where $N_{\rm veto}$ is the number of veto catalogs with non-detection (Gaia, 2MASS, Pan-STARRS, Legacy, NVSS, ROSAT). Maximum: 60 points.

\subsubsection{Thermal Score}

Based on effective temperature:
\begin{equation}
    S_{\rm thermal} = \begin{cases}
        20 & T_{\rm eff} < 300~\mathrm{K} \\
        10 & 300 \leq T_{\rm eff} < 400~\mathrm{K} \\
        5 & 400 \leq T_{\rm eff} < 500~\mathrm{K} \\
        0 & T_{\rm eff} \geq 500~\mathrm{K}
    \end{cases}
\end{equation}

\subsubsection{Kinematic Score}

Based on proper motion (as a proxy for distance):
\begin{equation}
    S_{\rm kinematic} = \begin{cases}
        -30 & \mathrm{parallax~detected~at~} 3\sigma \\
        -15 & \mu > 500~\mathrm{mas~yr}^{-1} \\
        0 & 100 < \mu < 500~\mathrm{mas~yr}^{-1} \\
        +10 & \mu < 100~\mathrm{mas~yr}^{-1}
    \end{cases}
\end{equation}
High proper motion objects are likely nearby brown dwarfs (natural explanation), while low proper motion may indicate more distant or extragalactic origin.

\subsubsection{Variability Score}

\begin{equation}
    S_{\rm variability} = \begin{cases}
        +25 & \mathrm{FADING} \\
        -10 & \mathrm{VARIABLE} \\
        +10 & \mathrm{NORMAL~(stable)}
    \end{cases}
\end{equation}
Stable thermal emission over a 10-year baseline is scored positively, while generic variability (common in astrophysical sources) is penalized. Systematic fading is highly unusual and receives the maximum bonus.

\subsection{Spectroscopic Follow-up Preparation}
\label{sec:followup}

For the highest-priority candidates, we prepare spectroscopic target lists including:

\begin{enumerate}
    \item Precise coordinates (J2000) with proper motion corrections to the observation epoch
    \item Finding charts from Legacy Survey DR10 imaging (when available)
    \item Visibility calculations for major near-infrared facilities (Keck/NIRES, VLT/KMOS, Gemini/GNIRS)
    \item Estimated exposure times based on W1 magnitude
\end{enumerate}

Spectroscopic observations in the 1--2.5~$\mu$m range would detect diagnostic features including CH$_4$ absorption (1.6, 2.2~$\mu$m) for Y/T dwarfs, H$_2$O bands (1.4, 1.9~$\mu$m), and NH$_3$ features characteristic of the coldest brown dwarfs.

\subsection{Software and Data Access}
\label{sec:software}

The TASNI pipeline is implemented in Python 3.10+ using:
\begin{itemize}
    \item \texttt{astropy} \citep{AstropyCollaboration2022} for coordinate transformations and time series analysis
    \item \texttt{astroquery} for catalog access via TAP/Cone Search protocols
    \item \texttt{pyvo} for Virtual Observatory services
    \item \texttt{pandas} for data manipulation
    \item \texttt{matplotlib} for visualization
\end{itemize}

Catalog queries are performed via:
\begin{itemize}
    \item IRSA TAP Service: \url{https://irsa.ipac.caltech.edu/TAP}
    \item Gaia Archive: \url{https://gea.esac.esa.int/archive}
    \item VizieR: \url{https://vizier.cds.unistra.fr}
    \item Legacy Survey: \url{https://www.legacysurvey.org}
\end{itemize}

\end{document}
