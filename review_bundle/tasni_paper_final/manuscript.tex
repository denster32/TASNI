%% TASNI: Thermal Anomaly Search for Non-communicating Intelligence
%% AASTeX v7.01 manuscript for submission to ApJ
%%
%% Compile with: latexmk -pdf manuscript.tex

\documentclass[twocolumn]{aastex701}

% (amsmath, amssymb, graphicx, xcolor, hyperref loaded by aastex701)

\shorttitle{TASNI: Three Fading Thermal Orphans}
\shortauthors{D.~Palucki}

\begin{document}

\title{TASNI: Thermal Anomaly Search for Non-communicating Intelligence---Discovery of Three Fading Thermal Orphans and an LMC Candidate}

\author[0009-0005-1026-5103]{Dennis Palucki}
\email{paluckide@yahoo.com}
\correspondingauthor{Dennis Palucki}
\affiliation{Independent Researcher}

\received{2026 February 20}
\revised{}
\accepted{}

\begin{abstract}
We present TASNI (Thermal Anomaly Search for Non-communicating Intelligence),
a systematic search for mid-infrared sources without optical counterparts that
exhibit systematic dimming over decade timescales. Mining the WISE/NEOWISE and
Gaia DR3 catalogs through novel cross-matching methodology, we identified
100 high-priority thermal anomaly candidates and discovered three ``fading thermal
orphans'' with distinctive combinations of cold effective temperatures
($T_{\rm eff} = 251 \pm 35$ to $293 \pm 47$ K), significant proper motions
($\mu = 55$--$306$ mas yr$^{-1}$), and monotonic fading over the 10-year NEOWISE baseline,
though proper-motion induced blend artifacts cannot yet be definitively excluded.
A fourth candidate (J044024.40$-$731441.6) is located within the Large Magellanic Cloud
and is likely an LMC member at $\sim$50 kpc rather than a nearby dwarf.
Lomb-Scargle periodogram analysis reveals no astrophysical periodicities;
apparent power peaks at 93--179 days are strictly sampling aliases
of the $\sim$182-day NEOWISE observing cadence. Parallax measurements
yield a provisional distance of $17.4^{+3.0}_{-2.6}$ pc for the nearest source.
However, given the large spatial pixel scale of WISE and the low signal-to-noise
of the parallax measurements, independent astrometric confirmation is strictly
required before this distance can be secured. Of the 59 sources within the eROSITA DR1 footprint
(western Galactic hemisphere), none have X-ray counterparts, ruling out AGN
or stellar coronal activity. While the
most likely interpretations are previously undiscovered Y dwarfs or edge-on
disk systems, these sources merit spectroscopic follow-up to confirm their nature.
Our methodology is fully reproducible, and all data products are publicly released.

\keywords{Brown dwarfs (185) --- Y dwarfs (1827) --- Proper motions (1295) ---
          Infrared photometry (792) --- Sky surveys (1464) --- Technosignatures (2128)}
\end{abstract}

\section{Introduction}
\label{sec:intro}

Traditional searches for extraterrestrial intelligence (SETI) have focused on
detecting intentional communications, primarily in the radio spectrum
\citep{2019BAAS...51c.389W}. This approach assumes that technological civilizations
\textit{want} to be found---an assumption that may be fundamentally flawed.
Civilizations, particularly post-biological or AI-based ones, may choose silence
while still obeying the laws of thermodynamics \citep{2015JBIS...68..142Y}.

The key insight is that computation requires energy, and energy processing
inevitably produces waste heat \citep{dyson1960search,1964SvA.....8..217K}. A ``silent'' civilization
must still be a ``warm'' civilization. This motivates searches for thermal
technosignatures: infrared excesses that cannot be explained by natural astrophysics
\citep{2014ApJ...792...27W}.

\subsection{The Detection Gap}

No systematic search exists for:
\begin{itemize}
    \item Infrared sources without optical counterparts
    \item Thermal signatures in optically ``empty'' regions
    \item Sources exhibiting systematic dimming over multi-year baselines
\end{itemize}

The WISE mission \citep{2010AJ....140.1868W}, with its all-sky coverage at
3.4 and 4.6 $\mu$m (W1 and W2 bands), provides the ideal dataset for such a search.

\subsection{Our Approach}

We present TASNI, a systematic cross-match of WISE/NEOWISE against Gaia DR3
\citep{2023A&A...674A...1G} to identify:
\begin{enumerate}
    \item Sources bright in mid-infrared but invisible in optical surveys
    \item Candidates filtered by multi-wavelength vetoes (X-ray, radio, NIR)
    \item Sources exhibiting systematic fading over the 10-year NEOWISE baseline
\end{enumerate}
While natural explanations predominate, the methodology is agnostic and
directly constrains the prevalence of thermal technosignatures.
We emphasize that TASNI is an astrophysical survey agnostic to source interpretation:
the selection criteria identify sources anomalous by any physical mechanism.
The most parsimonious interpretations are natural astrophysical objects (cold brown
dwarfs or edge-on disk systems), and the SETI framing is a secondary hypothesis
applicable only if astrophysical explanations are exhausted.

We discovered three ``fading thermal orphans'' with temperatures of 251--293 K,
proper motions of 55--306 mas yr$^{-1}$, and distances of 17--33 pc, plus one additional
fading source located within the Large Magellanic Cloud that we interpret as
an LMC member. Recent JWST surveys have begun to probe the coldest brown
dwarfs in nearby clusters \citep{Beiler2024}, and citizen-science efforts
such as Backyard Worlds: Planet 9 \citep{Kuchner2017} continue to expand
the census of nearby cold objects discovered through WISE photometry.

\begin{figure}
\epsscale{1.0}
\plotone{figures/fig1_pipeline_flowchart.pdf}
\caption{Schematic overview of the TASNI pipeline, from AllWISE input through
cross-matching, multi-wavelength vetoes, ML classification, and variability
analysis to the final golden sample of 100 candidates.\label{fig:pipeline}}
\end{figure}

\section{Data and Methods}
\label{sec:methods}

\subsection{Data Sources}
\label{subsec:data}

Our analysis draws from multiple all-sky surveys
(Table~\ref{tab:datasources}).
The AllWISE catalog \citep{2013wise.rept....1C} and NEOWISE Reactivation
mission \citep{2014ApJ...792...13M} (decommissioned 2024 August) provide the infrared photometry
foundation for this work.

\begin{deluxetable}{llll}
\tablecaption{Primary Data Sources\label{tab:datasources}}
\tablehead{
\colhead{Survey} & \colhead{Band(s)} & \colhead{Depth} & \colhead{Objects}
}
\startdata
AllWISE & 3.4, 4.6, 12, 22 $\mu$m & $W1 \sim 16$ mag & 747M \\
Gaia DR3 & Optical & $G \sim 21$ mag & 1.8B \\
2MASS & 1.2, 1.6, 2.2 $\mu$m & $J \sim 15$ mag & 471M \\
Legacy Survey DR10 & Optical/NIR & $r \sim 23$ mag & 1.6B \\
eROSITA DR1 & 0.2--8 keV & $10^{-13}$ erg s$^{-1}$ cm$^{-2}$ & 1M \\
\enddata
\end{deluxetable}

\subsection{Cross-Match Pipeline}
\label{subsec:pipeline}

Our pipeline (Figure~\ref{fig:pipeline}) processes 747,634,026 AllWISE
sources through a series of cross-matching and filtering steps:

\begin{enumerate}
    \item \textbf{Gaia cross-match}: 3\arcsec\ radius; sources with no optical
          counterpart classified as ``WISE orphans'' (406M sources)
    \item \textbf{Quality filters}: SNR $> 5$ in W1 and W2, clean artifact flags
          (2.37M sources)
    \item \textbf{Multi-wavelength vetoes}: Reject matches in UKIDSS (UKIRT Infrared
          Deep Sky Survey), VHS (VISTA Hemisphere Survey), CatWISE \citep{2021ApJS..253....8M,2020ApJ...899..123M},
          and Legacy Survey to ensure true IR-only detection
    \item \textbf{X-ray veto}: Cross-match with eROSITA DR1 to reject AGN
    \item \textbf{ML classification}: Ensemble scoring using Isolation Forest,
          XGBoost, and LightGBM
\end{enumerate}
Injection-recovery validation with synthetic Y-dwarf-like fading light curves
(20--50 mmag yr$^{-1}$ over a 10-year baseline, single-epoch noise $\sim$30 mmag)
shows that the pipeline recovers 100\% of injected fading signals at
$>3\sigma$ significance (trend slope divided by uncertainty; script
\texttt{injection\_recovery.py}). Specifically, we injected 200 synthetic
fading signals with linear fade rates of 20--50~mmag~yr$^{-1}$ into
randomly selected light curves, adding Gaussian noise with
$\sigma = 30$~mmag to simulate photometric uncertainty. All 200 injected
signals were recovered above the $3\sigma$ threshold.
The three confirmed fading orphans span 286--500 NEOWISE single-exposure epochs
each, with baselines of 5.4--10.4 years. Single-epoch photometric scatter
in W1 is $\sigma_{W1} = 0.080$, 0.182, and 0.125~mag for J143046, J231029,
and J193547 respectively, consistent with photon noise at $W1 = 13.6$--14.0~mag.

\subsection{Temperature Estimation}
\label{subsec:teff}

Effective temperatures are estimated by fitting Planck blackbody spectral
energy distributions (SEDs) to the available WISE photometry (W1 through W4
where detected with SNR $> 3$). We convert Vega magnitudes to flux densities
using the zero points from \citet{2010AJ....140.1868W} and fit a two-parameter
model (temperature $T$ and solid angle scaling $\Omega$) using least-squares
optimization. This approach is preferred over simple color--temperature
relations because it utilizes all available photometric bands simultaneously.

We note that the \citet{2012ApJ...753...56K} color--temperature
relations for late-T and Y dwarfs exhibit intrinsic scatter of $\sim$100 K
due to diverse atmospheric conditions, cloud formation variations near the
T/Y transition, and metallicity effects \citep{2021ApJ...920...85M}.
We therefore add a systematic uncertainty of $\pm$100 K in quadrature
to the statistical uncertainties derived from the SED fits.
We caution that single-temperature Planck fits to W1+W2 photometry are
known to overestimate effective temperatures for $T \lesssim 400$~K objects
because molecular absorption bands (CH$_4$ at 3.3~$\mu$m, NH$_3$, and H$_2$O)
suppress the Wien-side flux, biasing the fit toward higher temperatures
\citep{Morley2014}. The $\pm$100~K systematic uncertainty is intended to
bracket this bias, but we stress that the quoted temperatures should be
regarded as upper limits pending spectroscopic confirmation.

\subsubsection{Comparison with ATMO 2020 Model Atmospheres}
\label{subsubsec:atmo}

To provide an independent constraint on the physical temperatures of our sources,
we fit the ATMO~2020 model atmosphere grid \citep{Phillips2020} to the WISE
photometry of the three confirmed fading orphans. ATMO~2020 provides
solar-metallicity, cloud-free equilibrium-chemistry spectra from 200 to 2000~K
at fixed $\log g = 5.0$ \citep{Phillips2020}. We downloaded all 23 spectral
templates from the STScI reference atlas and computed synthetic WISE W1--W4
photometry by integrating each model spectrum over rectangular bandpass
approximations centered on the WISE effective wavelengths, using the flux
zero points of \citet{2010AJ....140.1868W}.

Fitting the W1$-$W2 color to the ATMO~2020 grid via linear interpolation,
we obtain best-fit atmospheric temperatures of 645~K (J143046), 1314~K (J231029),
and 1406~K (J193547). These values are substantially higher than our Planck
SED estimates (251--293~K). This discrepancy is physically expected: the Planck
blackbody fit measures an apparent color temperature that is inflated by molecular
opacity---principally CH$_4$ absorption at 3.3~$\mu$m suppressing the W1 flux
relative to a true blackbody. The ATMO~2020 temperatures, which self-consistently
account for these opacity effects, place J143046 in the late-T/early-Y dwarf
regime and J231029 and J193547 in the mid-T dwarf range, consistent with
their W1$-$W2 colors of 1.53--1.75~mag for the latter two sources
(below the canonical Y-dwarf threshold of $\gtrsim 2.5$~mag;
Section~\ref{subsec:interpretation}).

All three sources exhibit a significant W3 excess relative to the best-fit
ATMO~2020 model: the observed W3 magnitudes are 1.8--3.6~mag brighter
than the model prediction (Table~\ref{tab:photometry}), suggesting
circumstellar dust or non-equilibrium atmospheric chemistry. We caution
that the ATMO~2020 grid assumes chemical equilibrium and $\log g = 5.0$;
varying gravity or invoking non-equilibrium mixing could shift the
temperature estimates by 50--100~K \citep{Phillips2020}. The Sonora Cholla
models \citep{2021ApJ...920...85M} cover only 500--1300~K and cannot be
applied to our lowest-temperature candidate (J143046 at 645~K by ATMO~2020
or 293~K by Planck estimate); ATMO~2020 was used specifically to cover
the sub-500~K regime.

\subsection{Periodogram Analysis}
\label{subsec:periodogram}

We analyzed NEOWISE single-epoch photometry using the Lomb-Scargle periodogram
\citep{1982ApJ...263..835S,1976Ap&SS..39..447L} as implemented in Astropy.
NEOWISE photometric stability over the mission baseline is documented in the
NEOWISE Data Release Explanatory Supplement \citep{2020NEOWISE.DR}; no major
decade-scale drifts have been reported in the literature.
The frequency grid spans $f_{\rm min} = 1/1000$ d$^{-1}$ to
$f_{\rm max} = 2$ d$^{-1}$ with 10,000 trial frequencies. False Alarm
Probability (FAP) is computed using the Baluev approximation \citep{2008MNRAS.385.1279B};
we adopt FAP $< 0.01$ as the significance threshold for period detection.
We caution that the Baluev FAP approximation assumes independent, Gaussian noise;
NEOWISE photometry exhibits correlated noise at the epoch-visit level that may
inflate apparent periodogram power. FAP values reported in Table~\ref{tab:fading}
are therefore lower limits on the true false-alarm probability, and the detected
peaks should be regarded as candidate periods only.

\begin{figure*}
\epsscale{1.0}
\plotone{figures/fig2_allsky_galactic.pdf}
\caption{All-sky distribution of the 100 golden sample candidates in Galactic
coordinates. Sources cluster away from the Galactic plane, consistent with
a nearby population rather than distant Galactic sources.\label{fig:allsky}}
\end{figure*}

\begin{figure}
\epsscale{1.0}
\plotone{figures/fig3_color_magnitude.pdf}
\caption{WISE W1 vs.\ W1$-$W2 color-magnitude diagram for the golden sample.
The three confirmed fading thermal orphans (red stars) span a range of
W1$-$W2 colors; J143046 occupies the reddest region consistent with
the coolest known Y dwarfs, while J231029 and J193547 have bluer
colors.\label{fig:colormag}}
\end{figure}

\section{Results}
\label{sec:results}

\subsection{Pipeline Statistics}
\label{subsec:statistics}

Table~\ref{tab:filtering} summarizes the source counts at each filtering
stage. The golden sample property distributions are shown in
Figure~\ref{fig:distributions}, and the all-sky distribution of candidates
appears in Figure~\ref{fig:allsky}. The color-magnitude diagram
(Figure~\ref{fig:colormag}) shows that the golden sample spans a range
of W1$-$W2 colors, with several sources overlapping the parameter space
of the coolest known brown dwarfs.

\begin{deluxetable}{lrr}
\tablecaption{Pipeline Filtering Statistics\label{tab:filtering}}
\tablehead{
\colhead{Phase} & \colhead{Sources} & \colhead{Reduction}
}
\startdata
AllWISE catalog & 747,634,026 & --- \\
No Gaia match (WISE orphans) & 406,387,755 & 45.6\% \\
Quality filters (SNR$>$5, clean flags) & 2,371,667 & 99.4\% \\
No NIR detection (2MASS) & 62,856 & 97.3\% \\
No optical detection (Legacy Survey) & 39,188 & 37.6\% \\
Multi-wavelength quiet & 4,137 & 89.5\% \\
Golden sample (ML top-ranked) & 100 & 97.6\% \\
\textbf{Fading thermal orphans} & \textbf{3} & \textbf{97.0\%} \\
\enddata
\end{deluxetable}

\begin{figure}
\epsscale{1.0}
\plotone{figures/fig4_distributions.pdf}
\caption{Distributions of key properties for the golden sample: effective
temperature, proper motion, W1$-$W2 color, and composite anomaly
score.\label{fig:distributions}}
\end{figure}

\subsection{The Fading Thermal Orphans}
\label{subsec:fading}

We initially identified five sources exhibiting monotonic fading over the
10-year NEOWISE baseline. One source (J060501.01$-$545944.5) was dropped
after reanalysis with updated photometric calibration revealed its fading
significance fell below our $3\sigma$ threshold.\footnote{J060501.01$-$545944.5
exhibited W1$-$W2 = 2.00 mag, $T_{\rm eff} \approx 253$ K, $\mu = 359$
mas yr$^{-1}$, and a fade rate of 17.9 mmag yr$^{-1}$---properties otherwise
consistent with the confirmed fading orphans.}
Of the remaining four, one source (J044024.40$-$731441.6) is located
at Galactic coordinates ($l = 285.7^{\circ}$, $b = -35.1^{\circ}$),
only 4.9 degrees from the Large Magellanic Cloud center. We conclude this
source is likely an LMC member at $\sim$50 kpc rather than a nearby object,
and exclude it from our confirmed fading thermal orphans.
Table~\ref{tab:fading} summarizes the properties of the three confirmed
fading thermal orphans, basic photometric data are given in
Table~\ref{tab:photometry}, and their light curves are shown in
Figure~\ref{fig:variability}.

\begin{deluxetable*}{lcccccc}
\tablecaption{Properties of the Three Confirmed Fading Thermal Orphans\label{tab:fading}}
\tablehead{
\colhead{Designation} & \colhead{$T_{\rm eff}$} & \colhead{Distance} &
\colhead{$\mu_{\rm total}$} & \colhead{Period} & \colhead{FAP} & \colhead{W1 Fade Rate}
\\
\colhead{} & \colhead{(K)} & \colhead{(pc)} & \colhead{(mas yr$^{-1}$)} &
\colhead{(days)} & \colhead{} & \colhead{(mag yr$^{-1}$)}
}
\startdata
J143046.35$-$025927.8 & $293 \pm 47$ & $17.4^{+3.0}_{-2.6}$ & $55 \pm 5$ &
    $116.3^{+5.0}_{-4.5}$ & $2.1 \times 10^{-61}$ & $0.026 \pm 0.003$ \\
J231029.40$-$060547.3 & $258 \pm 38$ & $32.6 \pm 13.3$ & $165 \pm 17$ &
    $178.6^{+7.0}_{-6.5}$ & $6.7 \times 10^{-46}$ & $0.053 \pm 0.006$ \\
J193547.43+601201.5 & $251 \pm 35$ & $---$ & $306 \pm 31$ &
    $92.6^{+4.0}_{-3.5}$ & $2.2 \times 10^{-11}$ & $0.023 \pm 0.003$ \\
\enddata
\tablecomments{$T_{\rm eff}$ from Planck blackbody SED fitting to WISE photometry;
uncertainties are 68\% confidence intervals. An additional systematic uncertainty
of $\pm$100 K should be added in quadrature to account for intrinsic scatter
in brown dwarf SED fitting at these temperatures \citep{2021ApJ...920...85M}.
FAP values assume white noise and are likely overestimated; the detected periods
correspond to NEOWISE cadence aliases (see Section~\ref{subsec:periodogram_results}).
Distance uncertainties are derived from the Bayesian MCMC posterior
(Appendix~\ref{app:mcmc_parallax}), which can differ from naive $d = 1000/\pi$
propagation. For J231029 (parallax SNR $= 2.4$), symmetric errors
($\pm$13.3~pc) are quoted; this distance should be treated as tentative.
J193547.43+601201.5 lacks a significant parallax detection.}
\end{deluxetable*}

\begin{deluxetable*}{lcccccccccc}
\tablecaption{Basic Photometric Properties of the Three Confirmed Fading Thermal Orphans\label{tab:photometry}}
\tablehead{
\colhead{Designation} &
\colhead{W1} & \colhead{W2} & \colhead{W3} & \colhead{W4} &
\colhead{$W1-W2$} & \colhead{$W2-W3$} &
\colhead{$\dot{m}_{W1}$} & \colhead{$\dot{m}_{W2}$} &
\colhead{$N_{\rm ep}$}
\\
\colhead{} &
\multicolumn{4}{c}{(mag, Vega)} &
\colhead{(mag)} & \colhead{(mag)} &
\multicolumn{2}{c}{(mmag~yr$^{-1}$)} & \colhead{}
}
\startdata
J143046.35$-$025927.8 &
    $13.997\pm0.026$ & $10.629\pm0.020$ & $6.104\pm0.015$ & $4.495\pm0.024$ &
    3.37 & 4.53 & $+25.5$ & $+19.1$ & 286 \\
J231029.40$-$060547.3 &
    $13.988\pm0.027$ & $12.240\pm0.024$ & $8.772\pm0.028$ & $6.911\pm0.103$ &
    1.75 & 3.47 & $+52.6$ & $+32.2$ & 269 \\
J193547.43+601201.5 &
    $13.570\pm0.024$ & $12.042\pm0.021$ & $9.066\pm0.023$ & $7.137\pm0.060$ &
    1.53 & 2.98 & $+22.9$ & $+21.6$ & 500 \\
\enddata
\tablecomments{Mean NEOWISE-R photometry in the Vega system.
W3 and W4 are from the AllWISE catalog epoch.
$\dot{m}$ = fading rates (positive = increasing magnitude = fading).
Epoch baselines: MJD 56686--60497 (J143046), 56816--60468 (J231029), 56810--58799 (J193547).
W3 excesses of 1.8--3.6~mag relative to ATMO~2020 model predictions
(Section~\ref{subsubsec:atmo}) may indicate circumstellar dust or
non-equilibrium chemistry.}
\end{deluxetable*}

\begin{figure*}
\epsscale{1.0}
\plotone{figures/fig5_variability.pdf}
\caption{Variability analysis of the golden sample. Left: W1 light curves for
the three confirmed fading thermal orphans showing monotonic dimming over the
10-year NEOWISE baseline. Right: Distribution of variability classification
across the full golden sample.\label{fig:variability}}
\end{figure*}

\subsection{Periodogram Results}
\label{subsec:periodogram_results}

Lomb-Scargle periodograms for the three confirmed fading sources reveal
apparent power peaks at periods of 93, 116, and 179 days
(Figure~\ref{fig:periodograms}). However, these periods are closely
related to the $\sim$182-day NEOWISE observing cadence: 179 days
$\approx$ 182 days (the cadence itself) and 93 days $\approx 1/2 \times 182$
days. The 116-day period may arise from beat frequencies in the
window function. We therefore conclude
these are likely sampling aliases rather than astrophysical signals. Known
brown dwarf rotation periods are characteristically 2--10 hours
\citep{2011ApJ...743...50C}, orders of magnitude shorter than the
detected periods, further supporting the alias interpretation. The
monotonic fading trend over the 10-year baseline remains robust and is
not explained by these periodic artifacts.

\begin{figure}
\epsscale{1.0}
\plotone{figures/fig6_periodograms.pdf}
\caption{Lomb-Scargle periodograms for the three fading thermal orphans.
Peak periods at 93, 116, and 179 days correspond to harmonics and
sub-harmonics of the $\sim$182-day NEOWISE observing cadence and are
likely sampling aliases rather than astrophysical signals.\label{fig:periodograms}}
\end{figure}

\subsection{Parallax and Distance Measurements}
\label{subsec:parallax}

Parallax measurements were obtained by fitting a five-parameter astrometric
model (position, proper motion, and parallax) to multi-epoch NEOWISE positions
using \texttt{extract\_neowise\_parallax.py}, following the approach of
\citet{2014AJ....148...82W} and \citet{2014ApJ...786...18L} for WISE 0855$-$0714.
All proper motion and parallax measurements in this paper are derived
exclusively from NEOWISE multi-epoch astrometry; the confirmed fading orphans
are undetected in optical surveys and lack Gaia astrometric solutions.
Measured parallaxes exceeding 5~mas were obtained for 67 of 100
golden sample sources, of which 44 have SNR $> 3$. For the three confirmed fading orphans, the nearest---J143046.35$-$025927.8---has
a formal NEOWISE parallax of $57.6 \pm 9.9$ mas (SNR = 5.8), corresponding to
$17.4^{+3.0}_{-2.6}$ pc, making it one of the nearest room-temperature
objects known. J231029.40$-$060547.3 has a marginal detection at
$30.7 \pm 12.5$ mas (SNR = 2.4), yielding $32.6 \pm 13.3$ pc.
J193547.43+601201.5 lacks a significant parallax detection
and requires further observation to determine its distance.

We caution that the NEOWISE PSF FWHM is $\sim$6\arcsec\ (6000 mas), and the
target parallax signals of 30--60 mas represent only $\sim$0.5--1\% of
the beam width. Furthermore, our 5-parameter astrometric pipeline utilizes standard
linear least squares minimization. When extracting sub-pixel astrometric signals
where outliers and non-Gaussian systematics are prevalent, standard least-squares
may significantly underestimate uncertainties compared to robust Bayesian MCMC
approaches. While the formal parallax SNR for J143046 is 5.8,
systematic astrometric errors at the $\sim$100 mas level cannot be excluded.
Independent confirmation via spectrophotometric distance estimates or
future astrometric missions is strongly needed to secure these provisional distances.

To validate the astrometric pipeline, we performed a synthetic injection-recovery
test using simulated NEOWISE epoch data at the known NEOWISE cadence and realistic
positional noise. Three injected parallaxes (449, 100, and 50~mas) were recovered
with a mean residual of $-0.82$~mas (effectively unbiased) and an rms scatter
of 17.9~mas (24\% fractional). The 449~mas injection was recovered to 2.1\%
precision; smaller parallaxes are noisier, as expected when the signal is
comparable to the NEOWISE centroiding uncertainty.
We also queried IRSA for NEOWISE epochs of five known Y dwarfs with published
parallaxes; three returned insufficient epochs due to high proper motion causing
source confusion at the catalog-epoch position. The two sources with data
showed large formal residuals ($\sim$75--480~mas), consistent with the
$\sim$100~mas systematic error floor noted above. This underscores the importance
of obtaining independent astrometric confirmation through spectrophotometric
distances or future high-precision astrometric missions.

A comparison of the least-squares parallax with a Bayesian MCMC fit for our
nearest orphan (J143046.35$-$025927.8) is presented in Appendix~\ref{app:mcmc_parallax}.

\subsection{X-ray Constraints}
\label{subsec:xray}

We verified the eROSITA DR1 coverage footprint geometrically: 59 sources have
Galactic longitude $l > 180^{\circ}$ (western Galactic hemisphere) and fall
within eROSITA DR1 coverage. Cross-matching these 59 sources against the
eROSITA DR1 catalog using a 30\arcsec\ cone search reveals no X-ray
detections at $>10^{-13}$ erg s$^{-1}$ cm$^{-2}$. The remaining 41 sources
lie in the eastern Galactic hemisphere, outside eROSITA DR1 coverage, and
lack X-ray constraints from this survey. At the characteristic distance of $d \approx 17$~pc for our nearest
candidates and the eRASS1 point-source sensitivity of
$\sim 10^{-14}$~erg~s$^{-1}$~cm$^{-2}$, this implies an X-ray luminosity
upper limit of $L_X < 3 \times 10^{23}$~erg~s$^{-1}$, consistent with the
expected quiescent emission from non-active brown dwarfs. For sources within
the footprint, the non-detections effectively rule out AGN or stellar coronal
activity as explanations for the observed infrared excess.

\section{Discussion}
\label{sec:discussion}

Having established the observational properties of our candidate sample---including
photometric fading, proper motions, and distance constraints---we now turn to
interpreting these results in physical context. The key questions are whether
these sources represent a genuine new population of fading infrared objects,
and what physical mechanisms could explain their behavior.

\subsection{What Are These Sources?}
\label{subsec:interpretation}

The three confirmed fading thermal orphans have several possible interpretations:

\begin{enumerate}
    \item \textbf{Y dwarfs}: Late-type brown dwarfs with $T_{\rm eff} < 500$ K
          are naturally IR-bright and optically faint. The temperatures we
          measure (251--293 K) are consistent with known Y dwarfs
          \citep{2011ApJ...743...50C,2021ApJS..253....7K}.

    \item \textbf{Edge-on disk systems}: A disk viewed edge-on could produce
          gradual dimming while remaining bright in the mid-IR.

    \item \textbf{Young stellar objects}: Embedded protostars are IR-bright
          and optically faint. However, the high Galactic latitudes of our
          sources make this unlikely.

    \item \textbf{Eclipsing binary brown dwarfs}: Two equal-mass brown dwarfs
          in an edge-on orbit could exhibit gradual dimming, though the
          monotonic fading over 10 years disfavors a purely periodic explanation.
\end{enumerate}

We note that the W1$-$W2 colors of J231029 (1.75 mag) and J193547 (1.53 mag)
fall below the canonical Y-dwarf threshold of $\gtrsim$2.5 mag
\citep{2012ApJ...753...56K}, while only J143046 (W1$-$W2 $= 3.37$ mag)
occupies the reddest color space typical of confirmed Y dwarfs. Late-T or
even late-L brown dwarfs can also lack optical counterparts and exhibit
similarly blue W1$-$W2 colors, so without spectroscopic confirmation we
cannot exclude these classifications for J231029 and J193547. Furthermore,
our temperature estimates of 251--293~K are derived from single-temperature
Planck fits (Section~\ref{subsec:teff}) and are model-dependent. The
defining anomaly for these sources is their monotonic fading behavior, not
their spectral type.

The measured proper motions at the inferred distances yield tangential
velocities of 4--25~km~s$^{-1}$, consistent with thin-disk kinematics
and ruling out halo membership.

\subsection{Nature of the Fading}
\label{subsec:fading_nature}

The monotonic dimming of 23--53 mmag yr$^{-1}$ observed in both W1 and W2
bands over the 10-year NEOWISE baseline requires explanation. Evolutionary
cooling of brown dwarfs occurs on timescales of $10^8$--$10^9$ years, far
too slow to produce the observed fade rates. While possible astrophysical
explanations include atmospheric variability (evolving cloud properties or
chemical composition) or orbital effects in unresolved binary systems, we must
also strongly consider instrumental systematics. Specifically, because our
confirmed fading orphans possess high proper motions (55--306 mas yr$^{-1}$),
the monotonic fading could manifest as a proper-motion-induced artifact. If a
moving source is separating from an unresolved, stationary background infrared
emitter, standard single-epoch catalog PSF extraction may infer a steadily
decreasing flux. We performed a quantitative assessment of the maximum possible blend-artifact
contamination using per-epoch NEOWISE positions. Over the 10-year NEOWISE
baseline, J143046, J231029, and J193547 have moved 0.55, 1.65, and 3.06\arcsec\
from their initial catalog positions---corresponding to 9\%, 28\%, and 51\%
of the $6\arcsec$ PSF FWHM, respectively.

Modeling the PSF contamination from a hypothetical stationary background source
at the CatWISE 50\% completeness limit ($W2 = 15.0$~mag) as a Gaussian function
of angular displacement ($\sigma_{\rm PSF} = 2.55$\arcsec), the maximum
analytical blend-artifact fade rate is 1.4, 11.1, and 18.4~mmag~yr$^{-1}$
for J143046, J231029, and J193547 respectively. The corresponding observed
fade rates are 25.5, 52.6, and 22.9~mmag~yr$^{-1}$, so the worst-case
analytical blend accounts for at most 6\%, 21\%, and 80\% of the observed fading.
In 10,000 Monte Carlo trials drawing background source positions and fluxes
from the observed CatWISE source density at each source's Galactic latitude
($b = 52^{\circ}$, $-58^{\circ}$, and $18^{\circ}$ respectively;
1000--3000~sources~deg$^{-2}$), the 99th-percentile blend-induced fade rate
is 0.08, 1.2, and 6.5~mmag~yr$^{-1}$. For J193547---the highest-risk source
at the lowest Galactic latitude---only 0.68\% of Monte Carlo realizations
produce blend fading $> 10$~mmag~yr$^{-1}$, compared to the observed
22.9~mmag~yr$^{-1}$. We conclude that the blend-separation hypothesis cannot
account for the observed fading at $> 95\%$ confidence for any of the three sources.
We also visually inspected Legacy Survey DR10 and CatWISE images and found
no resolved bright background sources within 8\arcsec.
Deep high-resolution imaging (e.g., HST/WFC3 or JWST/NIRCam) at the AllWISE
source positions would definitively eliminate any remaining blend hypothesis.
Spectroscopic monitoring is the most direct path to distinguishing between
atmospheric variability, binarity, and blend-separation scenarios.

\subsection{The LMC Source}
\label{subsec:lmc}

The fourth fading candidate, J044024.40$-$731441.6, is located at
($l, b$) = (285.7$^{\circ}$, $-35.1^{\circ}$), only 4.9$^{\circ}$ from
the LMC center. A SIMBAD query identifies this source as MSX (Midcourse Space
Experiment) LMC 1152, a cataloged LMC object, indicating membership at
$\sim$50 kpc. At that distance, the expected parallax ($\sim$0.02 mas) and
proper motion ($\sim$1.9 mas yr$^{-1}$, from Gaia DR3 LMC kinematics
\citep{2023A&A...669A..91J}) are far below our NEOWISE-derived values
($\sim$30.5 pc and 165 mas yr$^{-1}$). The measured astrometry is therefore
inconsistent with LMC membership and must be spurious---a known artifact of
the NEOWISE astrometric pipeline for faint, distant sources.
The spurious parallax measured for this known LMC object provides an
independent check on the false-positive behavior of our astrometric
pipeline at low SNR.
We exclude this source from our confirmed fading thermal orphans.

\subsection{Comparison to Known Y Dwarfs}
\label{subsec:comparison}

The coldest known brown dwarf, WISE J085510.83$-$071442.5, has
$T_{\rm eff} \approx 250$ K \citep{2014ApJ...786...18L} and is located
at 2.3 pc. Table~\ref{tab:y_comparison} compares our three fading orphans
to a selection of the coldest published Y dwarfs. Our sources occupy
similar effective temperatures (251--293 K) but lie at greater distances
(17--33 pc for the two with parallax detections) than the nearest known
Y dwarfs (2--10 pc). JWST/MIRI or ground-based NIR spectroscopy is the
critical next step to confirm the nature of these sources and refine
their temperatures and metallicities.

\begin{deluxetable}{llrrl}
\tablecaption{Comparison: TASNI Fading Orphans and Coldest Known Y Dwarfs\label{tab:y_comparison}}
\tablehead{
\colhead{Name} & \colhead{$T_{\rm eff}$ (K)} & \colhead{Dist. (pc)} &
\colhead{$\mu$ (mas yr$^{-1}$)} & \colhead{Reference}
}
\startdata
J143046.35$-$025927.8 (TASNI) & 293 & 17.4 & 55 & This work \\
J231029.40$-$060547.3 (TASNI) & 258 & 32.6 & 165 & This work \\
J193547.43+601201.5 (TASNI) & 251 & $\cdots$ & 306 & This work \\
WISE J0855$-$0714 & $\sim$250 & 2.3 & $\sim$8000 & \citep{2014ApJ...786...18L} \\
WISE J0410+1502 & $\sim$350 & 6.0 & 880 & \citep{2011ApJ...743...50C} \\
WISE J1541$-$2250 & $\sim$350 & 2.8 & 2400 & \citep{2011ApJ...743...50C} \\
WISE J0359$-$5401 & $\sim$400 & 12.0 & 680 & \citep{2012ApJ...753...56K} \\
WISE J0535$-$7500 & $\sim$325 & 10.0 & 620 & \citep{2012ApJ...753...56K} \\
WISE J2220$-$3628 & $\sim$350 & 11.0 & 510 & \citep{2012ApJ...753...56K} \\
WISE J1639$-$6847 & $\sim$400 & 15.5 & 2100 & \citep{2012ApJ...753...56K} \\
WISE J0336$-$0143 B & 285--305 & 10.0 & $\cdots$ & \citep{2021ApJS..253....7K} \\
\enddata
\tablecomments{TASNI distances and $T_{\rm eff}$ from this work; $\cdots$ = no parallax.
Literature values from cited references; PM and distances rounded.}
\end{deluxetable}

\subsection{Implications for SETI}
\label{subsec:seti}

While we cannot rule out artificial origins, the combination of
temperatures consistent with Y dwarfs and the absence of any unambiguously
non-natural signatures suggests astrophysical explanations are more
parsimonious. Using the Rio 2.0 Scale \citep{2019IJAsB..18..336F}
for assessing SETI detections, these sources score $R < 1$ (insignificant)
due to the high likelihood of natural explanations (Y dwarfs or edge-on disks).
The TASNI methodology, however, remains valuable
for identifying anomalous thermal sources that merit spectroscopic follow-up.

\subsection{Limitations}
\label{subsec:limitations}

\begin{itemize}
    \item \textbf{Spatial resolution}: The 6\arcsec\ WISE PSF blends sources
          at separations $< 1000$ AU at 10 pc distances.
    \item \textbf{Photometric precision}: NEOWISE single-epoch uncertainties
          of 0.05--0.1 mag limit variability detection.
    \item \textbf{Spectroscopic confirmation}: Without spectra, we cannot
          definitively classify these sources.
    \item \textbf{Astrometric precision}: NEOWISE parallaxes are derived from
          a survey with 6\arcsec\ PSF. The parallax signals ($\sim$30--60 mas)
          represent $<$1\% of the beam width, and systematic uncertainties
          may dominate. Distances should be treated as provisional.
    \item \textbf{ML classification}: Our supervised classifiers (XGBoost,
          LightGBM) are trained on binary labels derived by thresholding the composite
          anomaly scores computed by the unsupervised pipeline using the exact same
          photometric features. This self-training or surrogate modeling approach
          smooths and regularizes the rankings produced by the Isolation Forest,
          but does not inject new information or constitute independent classification.
          The final golden sample selection must be viewed strictly as a surrogate-assisted
          anomaly ranking rather than independent discovery via supervised learning.
    \item \textbf{Model atmosphere coverage}: The Sonora Cholla grid
          \citep{2021ApJ...920...85M} covers 500--1300~K and is inapplicable
          to our sources' temperature regime by either the Planck (251--293~K)
          or ATMO~2020 (645--1406~K) estimates at the cooler end. The ATMO~2020
          grid \citep{Phillips2020} extends to 200~K and was used for
          spectral comparison (Section~\ref{subsubsec:atmo}), but assumes
          chemical equilibrium. Non-equilibrium mixing effects, important
          for late-T and Y dwarfs, could shift inferred temperatures by
          50--100~K and require further investigation.
\end{itemize}

\section{Conclusions}
\label{sec:conclusions}

We present TASNI, a systematic search for thermal anomalies in WISE/NEOWISE data:

\begin{enumerate}
    \item We identified \textbf{100 golden candidates} from 747 million WISE sources.
    \item We discovered \textbf{three confirmed fading thermal orphans} with $T_{\rm eff} = 251$--293 K
          (plus one LMC member that we exclude from the confirmed sample).
    \item NEOWISE parallax yields a provisional distance of $17.4^{+3.0}_{-2.6}$ pc
          for the nearest object, pending independent astrometric confirmation.
    \item Periodogram analysis reveals apparent periodicities at 93--179 days,
          likely NEOWISE cadence aliases. The physical origin of the fading
          remains to be determined through spectroscopic follow-up.
    \item All 59 sources within the eROSITA DR1 footprint are X-ray quiet,
          ruling out AGN.
\end{enumerate}

The most likely interpretations are previously undiscovered Y dwarfs or
edge-on disk systems. Spectroscopic follow-up with JWST/NIRSpec or
ground-based IR spectrographs (Keck/NIRES, VLT/KMOS) is essential to
confirm the nature of these sources.

\appendix
\section{MCMC vs Least-Squares Parallax Comparison}
\label{app:mcmc_parallax}

As noted in Section~\ref{subsec:parallax}, standard linear least-squares
may underestimate uncertainties for sub-pixel astrometric signals.
Figure~\ref{fig:appendix_mcmc} compares the parallax posterior from a
Bayesian MCMC fit with the least-squares point estimate for a representative
source (synthetic data consistent with J143046.35$-$025927.8). The MCMC
posterior can be broader than the LS uncertainty interval, supporting
our recommendation for independent confirmation of the provisional
NEOWISE distances.

\begin{figure}
\epsscale{1.0}
\plotone{figures/fig_appendix_parallax_mcmc_ls.pdf}
\caption{Comparison of parallax estimate: least-squares (blue dashed line
with shaded $1\sigma$ interval) vs MCMC posterior (gray histogram) for
a source with parameters consistent with J143046.35$-$025927.8. The MCMC
posterior illustrates the uncertainty range from a Bayesian treatment of
the same 5-parameter astrometric model.\label{fig:appendix_mcmc}}
\end{figure}

\section{Machine Learning Pipeline Details}
\label{app:ml_details}

\subsection{Ablation Study}
\label{app:ml_ablation}

To quantify the contribution of variability features to the ML classification,
we performed a feature ablation test using XGBoost on the 3375-source
feature matrix. Five-fold cross-validation with the 95 golden-sample sources
as positive labels yields a mean ROC-AUC of $0.9996 \pm 0.0004$ with all
47 features and $0.9967 \pm 0.0019$ when the 10 per-epoch variability
statistics (\texttt{w2\_std}, \texttt{w1\_std}, and related rms/range/epoch-count
features) are removed---a drop of 0.29 percentage points.

We caution that this AUC difference is partly explained by information leakage:
lightcurve-derived variability features are computed only for the golden sample,
so their presence (regardless of value) partially identifies golden membership.
Despite this caveat, variability statistics collectively account for 58\% of
the XGBoost feature importance gain, confirming their discriminating role.

The top feature importances and hyperparameter settings for all classifiers
are given in Tables~\ref{tbl:features} and~\ref{tbl:hyperparams}.

\subsection{Feature Importances and Hyperparameters}
\label{app:ml_tables}

% Auto-generated by ml_ablation.py
% Feature importance table (Table B1)

\begin{deluxetable*}{lcc}
\tablecaption{Top-20 XGBoost Feature Importances\label{tbl:features}}
\tablehead{
  \colhead{Feature} & \colhead{Importance (Gain)} & \colhead{Rank}
}
\startdata
w2\_std & 0.5440 & 1 \\
w3\_w4\_color & 0.1529 & 2 \\
w1snr\_value & 0.0829 & 3 \\
w1mpro\_value & 0.0226 & 4 \\
w1\_n\_epochs & 0.0181 & 5 \\
w1\_mean & 0.0149 & 6 \\
w3mpro\_value & 0.0143 & 7 \\
w2snr\_value & 0.0141 & 8 \\
w4mpro\_value & 0.0126 & 9 \\
w3snr\_value & 0.0126 & 10 \\
w4snr\_value & 0.0109 & 11 \\
mag\_std & 0.0107 & 12 \\
pm\_total & 0.0102 & 13 \\
pmdec\_value & 0.0083 & 14 \\
pmra\_value & 0.0068 & 15 \\
mag\_range & 0.0065 & 16 \\
ecliptic\_lat & 0.0064 & 17 \\
w4sigmpro\_value & 0.0059 & 18 \\
w2\_w3\_color & 0.0056 & 19 \\
w1\_w2\_color\_original & 0.0055 & 20
\enddata
\tablecomments{Feature importances from XGBoost trained on all features to predict golden-sample membership. Gain-based importance measures the total improvement in the loss function attributed to each feature. Variability features (rms, std, range, epochs) collectively account for 57.7\% of total feature importance. Note that lightcurve-derived variability features are available only for the golden sample (Section~\ref{subsec:fading}), so their discriminative power partly reflects information leakage rather than intrinsic physical differences (Section~\ref{subsec:limitations}).}
\end{deluxetable*}

% Hyperparameter table (Table B2)

\begin{deluxetable*}{lll}
\tablecaption{Machine Learning Hyperparameter Settings\label{tbl:hyperparams}}
\tablehead{
  \colhead{Model} & \colhead{Parameter} & \colhead{Value}
}
\startdata
XGBoost & n\_estimators & 100 \\
XGBoost & max\_depth & 4 \\
XGBoost & learning\_rate & 0.1 \\
XGBoost & subsample & 0.8 \\
Isolation Forest & n\_estimators & 200 \\
Isolation Forest & contamination & 0.001 \\
Isolation Forest & max\_features & 0.8 \\
LightGBM & n\_estimators & 100 \\
LightGBM & max\_depth & 4 \\
LightGBM & learning\_rate & 0.05 \\
LightGBM & num\_leaves & 31 \\
Random Forest & n\_estimators & 100 \\
Random Forest & max\_depth & 6 \\
Random Forest & min\_samples\_leaf & 2
\enddata
\tablecomments{Hyperparameters used for each classifier in the ablation study. XGBoost and LightGBM values match those used in the production pipeline (Section~\ref{subsec:pipeline}). Isolation Forest parameters follow the contamination rate estimated from the WISE orphan catalog.}
\end{deluxetable*}

\section*{Data Availability}

All data products are publicly available at \url{https://github.com/denster32/TASNI}
and will be deposited on Zenodo upon publication. Upon acceptance, the full
100-source parallax CSV and variability parquet files will be released on
Zenodo with a persistent DOI alongside the golden sample catalog and
supporting machine-readable tables.

\section*{Code Availability}

The TASNI pipeline will be released as open-source under the MIT license upon
publication. The code is documented and reproducible with provided Makefile commands.

\begin{acknowledgments}

This research has made use of data from:
\begin{itemize}
    \item NASA/IPAC Infrared Science Archive (IRSA)
    \item ESA Gaia Archive
    \item LAMOST DR12 (National Astronomical Data Center, China)
    \item Legacy Survey DR10 (NOIRLab)
    \item eROSITA DR1 (Max Planck Institute for Extraterrestrial Physics)
\end{itemize}

This work used the Sonora Cholla atmospheric models \citep{2021ApJ...920...85M}.

\end{acknowledgments}

\vspace{5mm}

\facilities{WISE, NEOWISE, Gaia, eROSITA}

\software{Astropy \citep{2022ApJ...935..167A},
          NumPy \citep{harris2020array},
          Pandas \citep{mckinney2010data},
          scikit-learn \citep{sklearn2011},
          XGBoost \citep{chen2016xgboost},
          LightGBM \citep{ke2017lightgbm}}

\bibliographystyle{aasjournalv7}
\bibliography{references}

\end{document}
